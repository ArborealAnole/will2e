\def\pkgname{minibox}

\RequirePackage{filecontents}

%%%%%%%%%1%%%%%%%%%2%%%%%%%%%3%%%%%%%%%4%%%%%%%%%5%%%%%%%%%6%%%%%%%%%7%%%%%%%%%
\begin{filecontents*}{minibox.sty}
%    \begin{macrocode}
\RequirePackage{expl3}
\ProvidesExplPackage
  {minibox}
  {2013/06/21}
  {0.2a}
  {Another type of box.}
%    \end{macrocode}
%
%    \begin{macrocode}
\bool_new:N \l_minibox_frame_bool
\keys_define:nn {minibox}
 {
  frame .choice: ,
  frame / true  .code:n = { \bool_set_true:N  \l_minibox_frame_bool } ,
  frame / false .code:n = { \bool_set_false:N \l_minibox_frame_bool } ,
  frame .default:n  = { true } ,
  
  l .code:n = { \tl_set:Nn \l_minibox_tabular_preamble_tl {l} } ,
  c .code:n = { \tl_set:Nn \l_minibox_tabular_preamble_tl {c} } ,
  r .code:n = { \tl_set:Nn \l_minibox_tabular_preamble_tl {r} } ,
  
  t .code:n = { \tl_set:Nn \l_minibox_tabular_valign_tl {t} } ,
  m .code:n = { \tl_set:Nn \l_minibox_tabular_valign_tl {c} } ,
  b .code:n = { \tl_set:Nn \l_minibox_tabular_valign_tl {b} } ,
  
  rule .dim_set:N = \l_minibox_rule_dim ,
  pad  .dim_set:N = \l_minibox_pad_dim  ,
 }
%    \end{macrocode}
%
%    \begin{macrocode}
\cs_new:Npn \miniboxsetup #1 { \keys_set:nn {minibox} {#1} }
\miniboxsetup {l,m,rule=\fboxrule,pad=\fboxsep}
%    \end{macrocode}
%
%    \begin{macrocode}
\newcommand\minibox[2][]
 {
  \group_begin:
  \keys_set:nn {minibox} {#1}
  \bool_if:NTF \l_minibox_frame_bool
   {
    \setlength\fboxrule{\l_minibox_rule_dim}
    \setlength\fboxsep{\l_minibox_pad_dim}
    \fbox
   } 
   { \use:n }
   {
    \use:x
     {
      \exp_not:N \begin{tabular}
        [\l_minibox_tabular_valign_tl]
        { @{} \l_minibox_tabular_preamble_tl @{} }
       \exp_not:n {#2}
      \exp_not:N \end{tabular}
     }
   }
  \group_end:
 }
%    \end{macrocode}
\end{filecontents*}
%%%%%%%%%1%%%%%%%%%2%%%%%%%%%3%%%%%%%%%4%%%%%%%%%5%%%%%%%%%6%%%%%%%%%7%%%%%%%%%




% Conditionally compile the documentation & generate the .ins file:
\providecommand\pstoolCompile{Y}
\if\pstoolCompile N
  \expandafter\endinput
\fi


\begin{filecontents*}{\pkgname.ins}
%&latex
\def\pstoolCompile{N}
\input minibox.tex
\csname@@end\endcsname
\end{filecontents*}




\makeatletter
\documentclass{ltxdoc}

\usepackage[rm,medium]{titlesec}

\usepackage{xcolor,booktabs}
\usepackage[colorlinks,linktocpage]{hyperref}

\usepackage{tocloft,varwidth,array}
\setcounter{tocdepth}{1}
\def\tocwidthA{0.45}
\def\tocwidthB{0.45}
\def\cftpartfont{\scshape}
\def\cftsecfont{\small}
\cftbeforesecskip=0pt
\def\cftpartleader{}
\def\cftpartafterpnum{\cftparfillskip}
\def\cftsecleader{}
\def\cftsecafterpnum{\cftparfillskip}

\let\pkg\textsf
\def\pkgopt#1{\texttt{[#1]}}

\def\PDF{\textsc{pdf}}
\def\PS{\textsc{ps}}
\def\DVI{\textsc{dvi}}
\def\EPS{\textsc{eps}}

\usepackage{\pkgname}
\usepackage[T1]{fontenc}
\usepackage{microtype}
\usepackage{lmodern}
\linespread{1.1}
\frenchspacing

\GetFileInfo{\pkgname.sty}
\begin{document}

\title{The \pkg{\pkgname} package}
\author{Will Robertson\\\texttt{wspr81@gmail.com}}
\date{\fileversion\qquad\filedate}

\maketitle

\section{Introduction}

It's sometimes useful to be able to stack text over lines
in a small box; this is similar to paragraph text broken over lines, but 
for small amounts of text when automatic line breaking is not required.
In other words, I'm looking for an \cmd\mbox\ that allows manual line breaks.
\begin{center}
\fbox{\vbox{\hbox{abcd}\hbox{efg}\hbox{h}}}
\end{center}
This sort of thing is a little awkward in plain \TeX\ and \LaTeX.
\begin{center}
|\vbox{\hbox{abcd}\hbox{efg}\hbox{h}}|\par
\end{center}

\section{The command \cs{minibox}}

This package defines the \cmd\minibox\DescribeMacro{\minibox}\ command
to write this more conveniently separately lines with \verb"\\".
Various options are allowed to control the alignment and whether to frame the box, shown in Table~\ref{opt}.

\begin{table}
\centering
\begin{tabular}{rl}
\toprule
\texttt{frame(=true)}  & Has a frame \\
\texttt{frame=false} & Has not a frame (default) \\
\texttt{rule=}\meta{dim} & Thickness of the rule (default \the\fboxrule) \\
\texttt{pad=}\meta{dim} & Space on the inside of the frame (default \the\fboxsep) \\
\midrule
\texttt{l} & Left-aligned text (default) \\
\texttt{c} & Centred text \\
\texttt{r} & Right-aligned text \\
\midrule
\texttt{b} & Align the box with the bottom line \\
\texttt{m} & Vertically centre the box (default) \\
\texttt{t} & Align the box with the top line \\
\bottomrule
\end{tabular}
\caption{Optional arguments for the \cs{minibox} command.}
\label{opt}
\end{table}

\subsection{Horizontal alignment of the text}
Here's an example adjusting the horizontal alignment.
\miniboxsetup{frame}
\begin{center}
\verb|\def\x{abcd\\efg\\h}|
\def\x{abcd\\efg\\h}
\begin{tabular}{ccc}
\verb"\minibox{\x}" &
\verb"\minibox[c]{\x}" &
\verb"\minibox[r]{\x}" \\
\minibox{\x} &
\minibox[c]{\x} &
\minibox[r]{\x}
\end{tabular}
\end{center}

\subsection{Vertical alignment of the box}
Here's an example adjusting the vertical alignment.
\begin{center}
\verb|\def\x{abcd\\efg\\h}|
\medskip

\def\x{abcd\\efg\\h}
\begin{tabular}{ccc}
\verb"xyz\minibox{\x}" &
\verb"xyz\minibox[b]{\x}" &
\verb"xyz\minibox[t]{\x}" \\
xyz\minibox{\x} &
xyz\minibox[b]{\x} &
xyz\minibox[t]{\x}
\end{tabular}
\end{center}

\subsection{Framing your box}
The boxes which are showed in these examples are not displayed by default; use the \verb"frame" option to make them appear:
\miniboxsetup{frame=false}
\begin{center}
\verb|\def\x{abcd\\efg\\h}|
\medskip

\def\x{abcd\\efg\\h}
\begin{tabular}{ccc}
\verb"\minibox{\x}" &
\verb"\minibox[frame]{\x}" &
\verb"\minibox[frame,rule=1pt,pad=0pt]{\x}" \\
\minibox{\x} &
\minibox[frame]{\x} &
\minibox[frame,rule=1pt,pad=0pt]{\x} \\
\end{tabular}
\end{center}
Negative values can be input for \texttt{pad}.
\begin{quote}\itshape
\LaTeX\ experts: while these padding and rule options are internally controlled by \cs{fboxsep} and \cs{fboxrule}, changing these variables will have no effect on \cs{minibox}'s behaviour. Use \cs{miniboxsetup} as described next to change these options globally.
\end{quote}

\subsection{Setting options}

Obviously, any combination of these options can be applied:
\begin{center}
\verb|\def\x{abcd\\efg\\h}|
\medskip

\def\x{abcd\\efg\\h}
\begin{tabular}{ccc}
\verb"x\minibox[frame]{\x}" &
\verb"x\minibox[frame,c,b]{\x}" \\
x\minibox[frame]{\x} &
x\minibox[frame,c,b]{\x}
\end{tabular}
\end{center}
\DescribeMacro{\miniboxsetup}
You can change the defaults of \cs{minibox} using this command.
\begin{center}
\verb|\def\x{abcd\\efg\\h}|
\medskip

\def\x{abcd\\efg\\h}
\begin{tabular}{ccc}
& \verb"\miniboxsetup{frame,r}" \\
\verb"\minibox{\x}" &
\verb"\minibox{\x}" \\
\minibox{\x} &
\miniboxsetup{frame,r}\minibox{\x}
\end{tabular}
\end{center}

\section{Licence}

This package is freely modifiable and distributable under the terms and conditions of the \LaTeX\ Project Public Licence, version 1.3c or greater (your choice). The latest version of
this license is available at: \url{http://www.latex-project.org/lppl.txt}. This work is maintained by \textsc{Will Robertson}.

\newpage
\part{Implementation}
\parindent=0pt
\DocInput{\pkgname.sty}
\end{document}
